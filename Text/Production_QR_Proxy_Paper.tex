\documentclass[11pt]{article}
\usepackage{graphicx}
\graphicspath{{/Users/justindoty/Documents/Research/Dissertation/Production_QR_Proxy/Code/}}
\usepackage{graphics}
\usepackage{amsfonts}
\usepackage{amsmath}
\usepackage{amstext}
\usepackage{tabularx}
\usepackage{mathrsfs}
\usepackage{subfigure}
\usepackage[dvipsnames]{xcolor}
\usepackage{lscape}
\usepackage{longtable}
\usepackage{bm}
\usepackage{bbm}
\usepackage{chngcntr}
\usepackage{setspace}
\usepackage{caption}
\usepackage{float}
\usepackage{multirow}
\usepackage{booktabs}
\usepackage{natbib}
\usepackage{fancyvrb}
\usepackage{enumitem}
\usepackage[multiple]{footmisc}
\newtheorem{assump}{Assumption}[section]
\newtheorem{lemma}{Lemma}[section]
\newtheorem{theorem}{Theorem}[section]


\usepackage[multiple]{footmisc}

%Set margins and text size
\setlength{\textwidth}{6.5in} \setlength{\textheight}{8.8in}
\setlength{\topmargin}{-0.5in}
\setlength{\oddsidemargin}{-0.01in}{}
\setlength{\parskip}{1.6mm}
\parskip=.06in
{}
%Some useful short-cuts
\def\argmax{\mathop{\rm arg\,max}}
\def\argmin{\mathop{\rm arg\,min}}



\setcounter{section}{0} % starts numbering section 1start_value=[0.498,-0.2874,0.4314,0.5444];

\setcounter{page}{1}



\usepackage{hyperref}
\hypersetup{
    colorlinks=true,
    allcolors=Violet
}
\begin{document}

\title{Heterogeneity in Firms: \\
A Proxy Variable Approach for Quantile Production Functions
%\footnote{}
}

\author{Justin Doty\thanks{Department of Economics, University of Iowa, S321 Pappajohn Business Building, 21 E Market St, Iowa City, IA 52242. Email: \href{mailto:justin-doty@uiowa.edu}{\texttt{justin-doty@uiowa.edu}}} and Suyong Song\thanks{Department of Economics and Finance, University of Iowa, W360 Pappajohn Business Building, 21 E Market St, Iowa City, IA 52242. Email: \href{mailto:suyong-song@uiowa.edu}{\texttt{suyong-song@uiowa.edu}}}
}

\date {\today}
\maketitle


\begin{abstract}
We propose a new approach to estimate production functions in which output elasticities are heterogeneous across the firm size distribution. 
This paper extends the proxy variable approach for estimating production functions to the conditional quantiles of firm production. Production function parameters are estimated in a simple two-stage approach which relies a location-shift assumption on unobserved productivity. We show that this method allows us to capture heterogeneity in output elasticities along the firm size distribution that may not be found in conditional mean estimates of \cite{Olley1996} and \cite{Levinsohn2003}. We provide small-sample evidence in a Monte Carlo study to show that this approach is robust compared to other production function estimators. The method is applied to firm and plant-level manufacturing data from the US, Chile, and Colombia.
\end{abstract}


\textit{Keywords:} Production functions, Heterogeneous elasticity, Quantile regression

\textit{JEL Classification:} C14, C36, D24


\pagenumbering{arabic}

\baselineskip25pt

%\singlespacing
\onehalfspacing
%\doublespacing

\section{Introduction}

Production function estimation is an ongoing and historical empirical research topic that links firm's input to output decisions. Identification of the output elasticities and consequently the distribution of firm-level productivity is constrained by endogeneity issues. This is because productivity is unobserved by the econometrician, but observed by the firm when making input decisions. 

A popular approach to address this issue is to introduce a proxy variable such as investment, made popular by \cite{Olley1996} (OP) or an intermediate input using \cite{Levinsohn2003} (LP) or \cite{Ackerberg2015} (ACF). These proxies are functions of a state variable such as capital and unobserved productivity. Under certain assumptions, this function is strictly increasing in its scalar unobserved productivity component. Inverting this function controls for unobserved productivity and the production function parameters can be estimated with a simple two-stage estimator.

While these methods have been useful in identifying the production function parameters and recovering consistent estimates of total factor productivity (TFP), resulting estimates may be biased if there is additional heterogeneity in production technology across firms. Thus, allowing for heterogeneous coefficients is one possible way to capture these differences. The literature on heterogeneous production functions is small relative to the empirical research using the homogeneous coefficient model, even though many empirical studies have found heterogeneity in firm behavior and decisions.\footnote{Some notable examples are \cite*{Kasahara2015}, \cite*{balat} and \cite*{Li2017} to name of few. Also \cite{Gandhi2020} who estimate a nonparametric production function and obtain heterogeneous estimates.} This is because estimating the homogeneous coefficient model by itself is very difficult due to the issue of unobserved productivity. 

In our approach, we allow for non-neutrality of the unobserved idiosyncratic production shock, but require the component of productivity that is anticipated by firms to be Hicks-neutral.  We use the proxy variable approach in this framework in order to control for the part of production unobservables that are correlated with input decisions. The literature on control function approaches for quantile regression models is still a developing area therefore it is not straightforward to estimate production functions by allowing for endogenous inputs and their heterogeneous coefficients.\footnote{See for example \cite{Chesher2003}, \cite{Ma2006}, \cite{Lee2007} and \cite{2009a}} We are not aware of any published paper which takes into account the endogeneity issue of production functions in the conventional quantile regression framework. We fill the gap in this paper by proposing an easy-to-implement estimator.

We show through simulation, that our proposed two-step estimator performs well when productivity is estimated using the static control function approaches of either \cite{Levinsohn2003} or \cite{Ackerberg2015}. The estimator is successful in capturing  both heterogeneous output elasticities and controlling for unobserved productivity. In our empirical application, we consider several popular firm and plant-level manufacturing datasets and compare our estimator to the ACF estimator. We show that heterogeneity in these estimates implies differences in other features of firm production, such as returns to scale, capital intensity and productivity.

The rest of the paper is organized as follows. Section \ref{litreview} reviews prior approaches for production function estimation and the interpretations of quantile production functions. Section \ref{ourmodel} introduces the econometric model and the proposed estimator. Section \ref{montecarlo} presents finite-sample behaviors of the estimator via Monte Carlo experiments and Section \ref{application} applies this estimator to US, Chilean, and Colombian manufacturing datasets. Section \ref{conclusion} concludes with directions for future research.

\section{Literature Review} \label{litreview}
\subsection{Production Function Estimation}

We briefly review the ACF (2015) procedure for estimating a \textit{value-added} production function (in logs) \footnote{We consider a value-added production function here to be consistent with the model we introduce in Section \ref{ourmodel}}.

\begin{equation}
y_{it}=(\beta_{0}+\omega_{it})+\beta_{k}k_{it}+\beta_{l}l_{it}+\eta_{it},
\end{equation}
where $y_{it}$ denotes value-added output for firm $i$ at time $t$, $l_{it}$ denotes labor input, $k_{it}$ denotes capital input, $\omega_{it}$ is unobserved productivity and $\eta_{it}$ denotes an iid shock to production.\footnote{We present the production function this way (with the constant and productivity in parenthesis) since the constant is not separately identified from the mean of productivity. This notation is consistent with our model we introduce later.}

To control for the correlation between $\omega_{it}$ and inputs $k_{it}$ and $l_{it}$, ACF introduces an intermediate input demand defined as
\begin{equation}
m_{it}=m_{t}(k_{it}, l_{it}, \omega_{it}),
\end{equation}
where the function $m$ is strictly increasing in $\omega_{it}$ for all $k_{it}$ and $l_{it}$. Productivity can then be expressed as
\begin{equation}
\omega_{it}=m_{t}^{-1}(k_{it}, l_{it}, m_{it}).
\end{equation}
Substituting this equation into the production function they obtain
\begin{equation}
y_{it}=\beta_{0}+\beta_{k}k_{it}+\beta_{l}l_{it}+m^{-1}_{t}(k_{it}, l_{it}, m_{it})+\eta_{it}=\Phi(k_{it}, l_{it}, m_{it})+\eta_{it}.
\end{equation}
The function, $\Phi_{t}(k_{it}, l_{it}, m_{it})$, is identified by the following first stage moment restriction
\begin{equation}
\mathbb{E}[\eta_{it}|\mathcal{I}_{it}]=\mathbb{E}[y_{it}-\Phi_{t}(k_{it}, l_{it}, m_{it})|\mathcal{I}_{it}]=0,
\end{equation}
where $\mathcal{I}_{it}$ denotes the firm's information at time $t$. The first stage estimate of $\Phi_{t}$ can be obtained by a local linear regression or a polynomial regression in $(k_{it}, l_{it}, m_{it})$. This step controls for the unobservable $\eta_{it}$.

A second stage moment restriction identifies the coefficient on capital and labor. Assume that productivity follows an auto-regressive process
\begin{equation}
\omega_{it}=\mathbb{E}[\omega_{it}|\omega_{it-1}]+\xi_{it}=g(\omega_{it-1})+\xi_{it}
\end{equation}
where $\xi_{it}$ denotes an innovation to productivity and satisfies $\mathbb{E}[\xi_{it}|\mathcal{I}_{it-1}]=0$.

Then, the production function parameters can be estimated from the moment restrictions corresponding to orthogonality assumptions on the productivity innovation, $\xi_{it}$ by expressing productivity as a function of the parameters $\beta_{k}$ and $\beta_{l}$
\begin{equation}
\mathbbm{E}\Bigg[\hat{\xi}_{it}(\beta_{k}, \beta_{l})\otimes
\begin{pmatrix}
k_{it}\\
l_{it-1}
\end{pmatrix}\Bigg]=0
\end{equation}
where $\hat{\xi}_{it}(\beta_{k}, \beta_{l})$ is constructed from the residuals of a regression of $(\widehat{\beta_{0}+\omega_{it}})=\hat{\Phi}_{t}(k_{it}, l_{it}, m_{it})-\beta_{k}k_{it}-\beta_{l}l_{it}$ on its lagged value for a hypothetical value of $(\beta_{k}, \beta_{l})$.  

\subsection{Production Functions and Quantile Regression}


A quantile regression framework for production functions may facilitate many different interpretations. Since this framework has been applied in the production frontier literature, we briefly review how this may be a natural interpretation and the limitations of such a model. A (stochastic) frontier (SFA) model of production proposed by \cite{Aigner1977} introduces statistical error into a frontier model. Frontier models assume firms firms deviate from an optimal frontier of production. The SFA model is typically written as
\begin{equation}
y_{i}=f(x_{i}, \beta)+\varepsilon_{i},
\end{equation}
where $\varepsilon_{i}=\eta_{i}-u_{i}$, $x_{i}$ are inputs to production and $\beta$ are the parameters. The error term $\eta_{i}$ denotes the statistical noise in the model such as measurement error and $u_{it}$ represents one-sided deviations from the production frontier. Estimates of $\beta$ are typically obtained using maximum likelihood which requires strong distributional assumptions on the error terms. \cite{Bernini2004} suggests quantile regression could be used to estimate the highest percentiles of the conditional output distribution as an estimate of the production frontier. \cite{Aragon2005} uses a characterization of a deterministic frontier that interprets production functions as being from a continuous interval $\tau\in[0,1]$ where $\tau=1$ corresponds to the efficient frontier. The difficulty with these approaches is choosing which quantile to estimate as the frontier and performing inference on these extremal quantiles.\footnote{See \cite{Chernozhukov2005a} on how to extend inference in quantile regression to extreme tails in the conditional distribution} Predicting differences between the frontier and a production process for a given firm is also complicated since the predicted error is a composite of technical inefficiency and noise. Other econometric issues, such as endogeneity of input choices with respect to inefficiency are difficult to incorporate in this framework. The purpose of this paper is not to debate the advantages and disadvantages of production frontier models so we leave this discussion for future studies and acknowledge the theoretical challenges of quantile frontiers and instead focus on the standard production function model.

There are two main challenges with implementing a quantile regression framework to the standard production function model. First, addressing the endogeneity of $\omega_{it}$ using traditional panel data methods or control function approaches for quantile models is still a developing field that faces many challenges such as incidental parameters or non-smooth criterion functions. Second, it is not straightforward to link firms output decisions to variations in the ex-post shock $\eta_{it}$ without imposing some strict structural model of firm production.

Addressing the first point, we will argue that recent quantile panel data and quantile IV models should not be used for the same reasons panel data and IV models are not used for the conditional mean model. Quantile panel data models allow for flexible interactions between unobserved heterogeneity and the quantiles of the conditional response function. Some well known approaches assume a time-invariant fixed effect such as \cite{Koenker2004}, \cite{Lamarche2010}, or \cite{Canay2011} which assumes the fixed effect shifts only the location of the conditional quantile function. This approach has two disadvantages. First, assuming the unobservable is time-invariant is restrictive. \cite{Griliches1986} show that a fixed effect for productivity, $\omega_{i}$, leads to low estimates of the capital elasticity. Second, with the exception of the \cite{Canay2011} estimator, the fixed effects of these models are incidental parameters so as the sample size grows, so does the number of parameters that need to be estimated which makes it computationally costly. An alternative to fixed effect estimation is to model the unobserved heterogeneity as a projection onto the observables plus a disturbance in the spirit of \cite{Chamberlain1984}. This correlated random effect (CRE) approach was used by \cite{Abrevaya2008} and variations of this type of estimator have been developed by other researchers. One issue with the CRE approach is that identification of the conditional quantile function is difficult because it now depends on the joint distribution of unobservables in the response function and the random effect.

Another alternative is to make use of valid instruments if they are available. The conventional argument for using input prices $p^{k}_{it}$ and $p^{l}_{it}$ as instruments is that they must be uncorrelated with the error term $\omega_{it}+\eta_{it}$ and correlated with input choices for capital and labor. Then one could use two-stage least squares to obtain consistent estimates of $\beta_{k}$ and $\beta_{l}$. This idea can be extended to quantile-IV models such as \cite{Chernozhukov2005}. In their identification arguments, one would need to strengthen assumptions to conditional independence as well as monotonicity of a quantile structural function (QSF) in $U_{it}=\omega_{it}+\eta_{it}$. Then if one writes the QSF for the production function as $y_{it}=Q(k_{it}, l_{it}, U_{it})$, their the model is identified using
\begin{equation}
P[y_{it}\leq Q(k_{it}, l_{it}, \tau)|k_{it}, l_{it}, p^{k}_{it}, p^{l}_{it}]=\tau, \quad \tau\in (0,1].
\end{equation}
We do not use this procedure since input prices may not have enough variation across firms and exogeneity can be violated if they capture input quality differences as shown by \cite{Griliches1986}. Therefore, it is not straightforward to develop an identification strategy that also leads to a computationally simple estimator. This is a common challenge in the quantile IV literature. However, we will show that a simple adaption of the \cite{Canay2011} estimator works well in practice.


\section{A Random Coefficient Production Function} \label{ourmodel}
We specify a \textit{value-added} production function as a random coefficient model:
\begin{equation} \label{pfrc}
    y_{it}=(\beta_{0}+\omega_{it})+\beta_{k}(\eta_{it})k_{it}+\beta_{l}(\eta_{it})l_{it}
\end{equation}
The variables in equation \eqref{pfrc} have the same interpretation as the ones we introduced in the LP model. In this specification we allow the output elasticities to be functionally dependent on the production shock $\eta_{it}$ while productivity $\omega_{it}$ maintains its additive separability. The model presented here assumes that $(\beta_{0}+\omega_{it})$ is only a location-shift of the conditional output distribution. We assume the constant in the production function does not vary over $\tau$ since the constant is not separately identified from the mean of productivity. The location-shift specification is crucial in our approach since $(\beta_{0}+\omega_{it})$ also appears in the conditional mean counterpart of equation \eqref{pfrc} so that if this object can be consistently estimated from popular proxy variable approaches such as OP, LP or ACF, then with additional restrictions from the quantile IV literature, the production function parameters $\beta_{k}(\tau)$ and $\beta_{l}(\tau)$ can also be consistently estimated.

We should also mention that a value-added specification in equation \eqref{pfrc} is non-trivial. Value-added production functions are common in the empirical literature, however the objects recovered from a value-added model such as the output elasticities and TFP can only be mapped to its gross-output counterpart under special structural production functions such as Leontief value-added. Since the production shock $\eta_{it}$ enters \eqref{pfrc} non-separably, it is difficult to recover gross-output objects from value-added since the latter is composed of the additional error term. One consequence of a value-added model is that estimates of productivity may be more disperse than gross-output estimates which controls for material inputs. Therefore, in our empirical application we interpret the elasticities and productivity with some caution.

 One advantage of the value-added model is that the rank of $\eta_{it}$ can be interpreted as the rank of firm size measured by output net of material inputs. This way, we are able to measure firm size by the value of the firm's contribution to output rather than the value of the units they sell. The value-added approach also avoids the non-identification results of \cite{Gandhi2020}. We leave the connection between this value-added production function and its possibly underlying gross-output production function as future research agenda.

 An interesting question that follows from the firm size interpretation is then what are the determinants of firm size? Controlling for productivity also controls for firm size determinants such as managerial ability so variation in firm size might be limited in our model. There is a large and historical literature on the determinants of firm size. \cite{Kumar1999} survey some of the theories and test them using data. We offer a possible explanation below, but we focus on whether there is heterogeneity in firm size and its consequences. 

A special case of \eqref{pfrc} is the location-scale model,

\begin{equation} \label{locationscale}
    y_{it}=(\beta_{0}+\omega_{it})+\beta_{k}k_{it}+\beta_{l}l_{it}+(\mu_{k}k_{it}+\mu_{l}l_{it})\eta_{it}
\end{equation}
Which implies that the $\tau$-th conditional quantile of $y_{it}$ is given by

\begin{equation}
Q_{y_{it}}(\tau|\mathcal{I}_{it})=(\beta_{0}+\omega_{it})+\beta_{k}k_{it}+\beta_{l}l_{it}+(\mu_{k}k_{it}+\mu_{l}l_{it})F^{-1}(\tau)
\end{equation}
where $F^{-1}(\tau)$ is the quantile function of production shocks $\eta_{it}$.

The idea that input choices can impact firm's production beyond the conditional mean has important consequences for firm's attitude towards production risk. A volume of literature that originated in the late 1970's challenged the standard stochastic specifications of production functions \citep{Just1978,Just1979} by considering a specification that allowed firm's inputs to both increase or decrease the marginal variability of final output. These models are commonly applied to studying the agricultural industry where the variance on the yield of harvested crops could be increased by adverse weather or decreased by pesticide usage. Since manufacturing businesses tend to operate in a more controlled environment, risk is less prevalent in these industries so the conditional variance of $\eta_{it}$ may be smaller.

The connection between production risk and heterogeneity in quantiles is interesting so we briefly mention that under quantile preferences, a firm who maximizes the $\tau$ level of utility of profits could explain heterogeneity in the output distribution. Unlike risk-neutral firms, firms could have a utility function that is represented by preferences of the firm manager(s) who decides the optimal expenditure on inputs. Different managers may have different preferences for risk. A short list of papers have considered quantile utility maximization such as \cite{Manski1988}, \cite{ROSTEK2009}, \cite{Chambers2007}, and \cite{Bhattacharya2009}. Dynamic input choices such as investment are much more difficult to solve using the quantile utility framework and the reader can refer to \cite{Castro2017} for a treatment of dynamic quantile utility models. As far as we know, the quantile utility framework has not been applied to firm decision problems and a more thorough treatment of such is outside the scope of this paper.

A general quantile model such as the one specified in \eqref{pfrc} can be seen as an extension of the higher-order moment estimation of risk initiated by \cite{Antle1983}. However, it can also be seen purely as an econometric specification issue as we are unaware of any tests that could distinguish between higher order moment production risk and misspecification. We choose the latter interpretation for our model. 

\subsection{A Two-Step Estimator for Quantile Production Functions}
Let $\varepsilon_{it}=k_{it}[\beta_{k}(\eta_{it})-\beta^{\mu}_{k}]+l_{it}[\beta_{l}(\eta_{it})-\beta^{\mu}_{l}]$ where $(\beta_{k}^{\mu}, \beta_{l}^{\mu})=\boldsymbol{\beta}^{\mu}=\mathbbm{E}[\boldsymbol{\beta}(\eta_{it})]$ is the mean of the random coefficients. The conditional mean version of the random coefficient production function can then be written as
\begin{equation}
y_{it}=(\beta_{0}+\omega_{it})+\beta_{k}^{\mu}k_{it}+\beta_{l}^{\mu}l_{it}+\varepsilon_{it},
\end{equation}
where $\mathbbm{E}[\varepsilon_{it}|\mathcal{I}_{it}]=0$. This identity is useful since we can now use a consistent estimator of $(\beta_{0}+\omega_{it})$ which can be plugged into the random coefficient production function in equation \eqref{pfrc}. For example, if one is willing to make the appropriate assumptions regarding the timing of labor decisions and the structure of the proxy variable mentioned earlier, then $(\widehat{\beta_{0}+\omega_{it}})=\hat{\Phi}_{t}(k_{it}, m_{it})-\hat{\beta}_{k}^{\mu}k_{it}$ in the LP procedure. The ACF estimator can likewise be implemented by using  $(\widehat{\beta_{0}+\omega_{it}})=\hat{\Phi}_{t}(k_{it}, l_{it}, m_{it})-\hat{\beta}_{k}^{\mu}k_{it}-\hat{\beta}_{l}^{\mu}l_{it}$ as the labor elasticity is estimated in the second stage of their procedure.

Intuitively, the appropriate choice for estimation of $(\beta_{0}+\omega_{it})$ places an identification restriction on the location of the conditional distribution of $S_{t}=\beta_{k}(\eta_{it})k_{it}+\beta_{l}(\eta_{it})l_{it}$. \cite{Canay2011} approaches identification more formally. He shows using a deconvolution type argument that the distribution of the random coefficient component of the response variable is identified up to a location. The location of the conditional distribution in his model is identified from regularity conditions in a panel data model with first-differences of an unobserved individual effect. It is not straightforward how to extend this idea to our model where the unobserved effect, productivity, varies over time and exhibits serial correlation. It would be interesting to explore whether a flexible identification strategy for $S_{t}$ could be proposed such that the location of the conditional distribution can be identified under alternative production function approaches. For example, one may be interested in utilizing the methodology of \cite{Gandhi2020} to estimate productivity using an average gross-output production function. We leave this as future research agenda.

Once estimates of $(\beta_{0}+\omega_{it})$ are obtained, estimates of $\beta_{k}(\tau)$ and $\beta_{l}(\tau)$
can be obtained from a quantile regression of $\hat{y}_{it}=y_{it}-(\beta_{0}\hat{+}\omega_{it})$ with a constant omitted in estimation. To summarize these steps:
\begin{enumerate}
	\item Let $\hat{\beta}_{k}^{\mu}$ and $\hat{\beta}_{l}^{\mu}$ be consistent estimators of $\beta_{k}^{\mu}$ and $\beta_{l}^{\mu}$ from a value-added production function. For example, using the OP, LP, or ACF estimation strategy. Construct the estimator, $(\widehat{\beta_{0}+\omega_{it}})$ using these estimates
	\item Let $\boldsymbol{\beta}(\tau)=(\beta_{k}(\tau), \beta_{l}(\tau))$ and $\hat{y}_{it}=y_{it}-(\widehat{\beta_{0}+\omega_{it}})$. Define the two-step estimator of $\beta_{k}(\tau)$ and $\beta_{l}(\tau)$ as:
	\begin{equation*}
	\hat{\boldsymbol{\beta}}(\tau)=\underset{\boldsymbol{\beta}\in\mathcal{B}}{\operatorname{argmin}}\, \mathbbm{E}\big[\rho_{\tau}(\hat{y}_{it}-\beta_{k}k_{it}-\beta_{l}l_{it})\big],
	\end{equation*}
	where $\mathcal{B}$ is a compact and convex parameter space, $\rho_{\tau}(u)=u[\tau-\mathbbm{I}\{u<0\}]$ and $\mathbbm{I}\{\cdot\}$ denotes the indicator function.
\end{enumerate}

\section{Monte Carlo Experiments} \label{montecarlo}
In this section we investigate the performance of our estimator in a set of Monte Carlo experiments. We use a location-scale model for production given by Equation \eqref{locationscale} which we provide again in Equation \eqref{simmodel} below.

\begin{equation} \label{simmodel}
y_{it}=(\beta_{0}+\omega_{it})+\beta_{k}k_{it}+\beta_{l}l_{it}+(\gamma_{k}k_{it}+\gamma_{l}l_{it})\eta_{it},
\end{equation}
where $\beta_{0}=0, \beta_{k}=0.4$ and $\beta_{l}=0.6$. We replicate \cite{Ackerberg2015} simulations by sampling 1000 datasets consisting of 1000 firms. We simulate optimal input choices for 100 time periods, using the last 10 periods for estimation. We consider three different data generating processes (DGPs) for the scale parameters and the distribution of $\eta_{it}$. For DGP1, we let $\gamma_{k}=0.6$, $\gamma_{l}=-0.6$ and $\eta_{it}\sim N(0,\sigma_{\eta}^{2})$ where $\sigma_{\eta}^{2}=0.01$. This follows the original specification for the error term in \cite{Ackerberg2015}. In DGP2, $\gamma_{k}=0.4$, $\gamma_{l}=-0.4$ and $\eta_{it}\sim (\sqrt{3\sigma_{\eta}^{2}/5})t_{5}$ so that the error variance is the same as that of DGP1. In DGP3, we add considerable skewness to the ex-post shocks by letting $\gamma_{k}=0.5$, $\gamma_{l}=-0.5$ and $\eta_{it}\sim Lognormal(0.15,\sigma_{\eta}^{2})$. 

We show that our estimator is flexible using previously proposed proxy variable estimators. It should only matter that $\hat{\beta}_{k}^{\mu}$ and $\hat{\beta}_{l}^{\mu}$ can be identified and consistently estimated. It is well known that under certain timing assumptions, that the labor elasticity may not be identified using the first stage of \cite{Levinsohn2003}. Clearly, this would affect our estimates of productivity which we use in the second stage of our estimator. Therefore, we follow \cite{Ackerberg2015} in considering two DGPs where labor can and cannot be identified\footnote{Refer to \cite{Ackerberg2015} for a detailed discussion on how to generate the simulated data.}. In the DGPs prefixed by ``LP'' in Table \ref{simresults}, we add optimization error to labor which breaks the functional dependence problem argued by ACF. Therefore, LP and ACF estimates of labor and hence productivity, should be consistent. We use the LP estimator for this DGP. In the DGPs prefixed by ``ACF'' we add serially correlated wages and allow for labor to be chosen in a sub-period $t-b$, where $b=0.5$. In this setup, we know that the ACF estimates will be consistent since their estimator conditions on labor in the first stage whereas LP does not. Therefore, we use the ACF estimator for this DGP. An AR(1) process is specified for productivity $\omega_{it}=\rho\omega_{it-1}+\xi_{it}$ where $\rho=0.7$. The variance of $\xi_{it}$ and initial value $\omega_{i0}$ is set so that the standard deviation of $\omega_{it}$ is constant over time and equal to $0.3$. 

Table \ref{simresults} provides estimates of the bias and mean-squared error (MSE) for each DGP of our proposed estimator (DS) and naive quantile regression (QR) that does not control for simultaneity bias. Our estimator performs very well. The bias of our estimator remains low for each $\tau$ under each DGP for the error distribution and each DGP under LP and ACF estimates for productivity. The MSE is also small and reaches its minimum at the median ($\tau=0.5$). Not surprisingly, the QR estimates do not perform well in any situation. Both the bias and MSE are large compared to our estimates and the sign of the bias is positive for capital estimates and negative for labor estimates.

\pagebreak
\newpage

\begin{table}[H]
\centering
\caption{Monte Carlo Results for Capital and Labor Estimates}
\begin{tabular}{cccccccccc}
  \hline\hline & & & \multicolumn{2}{c}{DS} & & & \multicolumn{2}{c}{QR} \\ \cmidrule(lr){3-6} \cmidrule(lr){7-10} \\ & & \multicolumn{2}{c}{Capital} & \multicolumn{2}{c}{Labor} & \multicolumn{2}{c}{Capital} & \multicolumn{2}{c}{Labor} \\ \cmidrule(lr){3-4} \cmidrule(lr){5-6} \cmidrule(lr){7-8} \cmidrule(lr){9-10}DGP & $\tau$ & Bias & MSE & Bias & MSE & Bias & MSE & Bias & MSE \\ 
  \hline
LP 1 & 0.10 & -0.0149 & 0.0004 & 0.0199 & 0.0004 & 0.2691 & 0.0830 & -0.2771 & 0.0865 \\ 
   & 0.25 & -0.0045 & 0.0002 & 0.0065 & 0.0000 & 0.2905 & 0.0971 & -0.2975 & 0.1008 \\ 
   & 0.50 & 0.0010 & 0.0002 & 0.0000 & 0.0000 & 0.3140 & 0.1143 & -0.3210 & 0.1183 \\ 
   & 0.75 & 0.0065 & 0.0002 & -0.0065 & 0.0000 & 0.3375 & 0.1329 & -0.3445 & 0.1380 \\ 
   & 0.90 & 0.0169 & 0.0005 & -0.0199 & 0.0004 & 0.3589 & 0.1513 & -0.3659 & 0.1560 \\ 
  LP 2 & 0.10 & -0.0097 & 0.0003 & 0.0127 & 0.0002 & 0.2833 & 0.0610 & -0.2903 & 0.0635 \\ 
   & 0.25 & -0.0035 & 0.0002 & 0.0035 & 0.0000 & 0.2985 & 0.0870 & -0.3065 & 0.0905 \\ 
   & 0.50 & 0.0000 & 0.0002 & 0.0000 & 0.0000 & 0.3140 & 0.1149 & -0.3210 & 0.1190 \\ 
   & 0.75 & 0.0035 & 0.0002 & -0.0035 & 0.0000 & 0.3285 & 0.1460 & -0.3365 & 0.1506 \\ 
   & 0.90 & 0.0097 & 0.0003 & -0.0127 & 0.0002 & 0.3447 & 0.1849 & -0.3517 & 0.1910 \\ 
  LP 3 & 0.10 & -0.0053 & 0.0002 & 0.0083 & 0.0001 & 0.2857 & 0.0975 & -0.2927 & 0.1013 \\ 
   & 0.25 & -0.0015 & 0.0002 & 0.0035 & 0.0000 & 0.2935 & 0.0998 & -0.3005 & 0.1036 \\ 
   & 0.50 & 0.0004 & 0.0002 & -0.0004 & 0.0000 & 0.3064 & 0.1056 & -0.3144 & 0.1096 \\ 
   & 0.75 & 0.0049 & 0.0002 & -0.0059 & 0.0000 & 0.3269 & 0.1196 & -0.3349 & 0.1245 \\ 
   & 0.90 & 0.0167 & 0.0005 & -0.0197 & 0.0004 & 0.3537 & 0.1459 & -0.3617 & 0.1513 \\ 
   \hline
ACF 1 & 0.10 & -0.0099 & 0.0003 & 0.0129 & 0.0003 & 0.2881 & 0.0831 & -0.2941 & 0.0865 \\ 
   & 0.25 & -0.0035 & 0.0003 & 0.0045 & 0.0001 & 0.3115 & 0.0971 & -0.3175 & 0.1008 \\ 
   & 0.50 & -0.0010 & 0.0002 & 0.0000 & 0.0001 & 0.3380 & 0.1143 & -0.3440 & 0.1184 \\ 
   & 0.75 & 0.0025 & 0.0003 & -0.0035 & 0.0001 & 0.3645 & 0.1329 & -0.3715 & 0.1380 \\ 
   & 0.90 & 0.0089 & 0.0003 & -0.0119 & 0.0002 & 0.3889 & 0.1513 & -0.3949 & 0.1560 \\ 
  ACF 2 & 0.10 & -0.0057 & 0.0002 & 0.0087 & 0.0002 & 0.3053 & 0.0611 & -0.3103 & 0.0635 \\ 
   & 0.25 & -0.0015 & 0.0002 & 0.0025 & 0.0001 & 0.3225 & 0.0870 & -0.3285 & 0.0906 \\ 
   & 0.50 & 0.0000 & 0.0002 & 0.0000 & 0.0001 & 0.3390 & 0.1150 & -0.3450 & 0.1190 \\ 
   & 0.75 & 0.0025 & 0.0002 & -0.0025 & 0.0001 & 0.3555 & 0.1461 & -0.3615 & 0.1506 \\ 
   & 0.90 & 0.0067 & 0.0003 & -0.0077 & 0.0001 & 0.3727 & 0.1850 & -0.3787 & 0.1910 \\ 
  ACF 3 & 0.10 & -0.0043 & 0.0003 & 0.0043 & 0.0001 & 0.3067 & 0.0975 & -0.3127 & 0.1013 \\ 
   & 0.25 & -0.0015 & 0.0003 & 0.0015 & 0.0001 & 0.3145 & 0.0998 & -0.3205 & 0.1036 \\ 
   & 0.50 & -0.0006 & 0.0003 & 0.0006 & 0.0001 & 0.3294 & 0.1057 & -0.3354 & 0.1096 \\ 
   & 0.75 & 0.0019 & 0.0003 & -0.0029 & 0.0001 & 0.3519 & 0.1196 & -0.3579 & 0.1245 \\ 
   & 0.90 & 0.0097 & 0.0004 & -0.0137 & 0.0003 & 0.3817 & 0.1459 & -0.3877 & 0.1513 \\ 
   \hline
\end{tabular}
\caption*{1000 replications; ``DS'' is our estimator; ``QR'' is quantile regression with no correction for simultaneity bias; ``LP'' uses estimates of productivity from \cite{Levinsohn2003} under the DGPs that are favorable to their procedure;``ACF'' uses estimates of productivity from \cite{Ackerberg2015} under the DGPs that are favorable to their procedure}
\label{simresults}
\end{table}
\pagebreak
\newpage

\section{Application} \label{application}
We apply our estimator to popular firm and plant level manufacturing datasets from the US, Chile, and Colombia to examine heterogeneity in the output distribution.\footnote{We thank Mert Demirer for providing the datasets from Chile and Colombia} For each country, we examine estimates across different manufacturing industries as well as how these estimates have changed over time. We use the DS estimator presented in this paper and compare it to the LP estimates. In the case where timing assumptions on labor matter for which proxy variable approach is used to estimate productivity, we provide ACF estimates in the Appendix. We also compare our estimates to the quantile regression estimates without controlling for productivity. In the LP estimation procedure, we estimate the labor coefficient and $\Phi$ with a 3rd degree polynomial with interactions in capital and materials. To estimate capital, we use the LP criterion function mentioned earlier. Since we only need a consistent estimator of the production function parameters, we do not consider any over-identification conditions in this step. We use bootstrap to estimate standard errors of $\beta_{k}(\tau)$ and $\beta_{l}(\tau)$ with the number of iterations set to $500$.
\subsection{US Compustat}
The source for the US manufacturing data is from Compustat which covers publicly traded firms and contains data from their financial statements. We collect a sample between 1961 and 2010 on sales, capital expenditures, number of workers, and other expenses to construct measures of output, capital, labor, and material inputs using 3-digit deflators from \cite{nber}. Data preparation follows \cite{Keller2009} and \cite{mert}. Some issues regarding the Compustat dataset is that since the data is reported in the firm's financial statements, deflated output and input measures may not completely capture firm's actual usage. Also, since this sample only contains publicly traded firms, it is only a fraction of all manufacturing firms in the US. Summary statistics for these deflated values are provided in Table \ref{USsum} . We present a series of output elasticity estimates in Table \ref{USestACF} which are illustrated graphically in Figures \ref{fig:ACF31coef}, \ref{fig:ACF32coef}, \ref{fig:ACF33coef}, and \ref{fig:ACFUSallcoef}.

Estimates of the capital elasticity are increasing in the firm size distribution and nearly double from $\tau=0.1$ to $\tau=0.9$ in every industry as well as the combined sample.  The estimates for labor elasticity for each industry and the combined sample are decreasing. For NAICS 33, the labor elasticity increases at very small $\tau=0.1$, but then decreases. There is evidence that our model captures some heterogeneity compared to the LP model. In NAICS 32, there is more heterogeneity in the tails of the firm size distribution for capital estimates, but only the labor estimates are lower than the LP estimate for very large $\tau$. The same relationship in these estimates is true in NAICS 33 and in the combined sample. 

In each industry we compare the differences between DS and QR estimates to test whether our model corrects for endogeneity from unobserved productivity. Bootstrap is used to construct confidence intervals of the difference between the two estimates. In NAICS 31, only the capital estimates appear to be different from the QR estimates. For every other industry, both capital and labor estimates are different from QR estimates. In NAICS 32, as $\tau$ increases, the differences between our estimator and the QR estimator become larger. The opposite is true for NAICS 33 and the combined sample.
\begin{figure}[H]
\centering
\caption{Estimated Coefficients of Capital and Labor for US: NAICS 31}
\includegraphics[width=9cm, height=9cm]{/Users/justindoty/Documents/Research/Dissertation/Production_QR_Proxy/Code/Empirical/US/Plots/QACF_Coef_Plot_NAICS_31.png}
\label{fig:ACF31coef}
\end{figure}

\begin{figure}[H]
\centering
\caption{Estimated Coefficients of Capital and Labor for US: NAICS 32}
\includegraphics[width=9cm, height=9cm]{/Users/justindoty/Documents/Research/Dissertation/Production_QR_Proxy/Code/Empirical/US/Plots/QACF_Coef_Plot_NAICS_32.png}
\label{fig:ACF32coef}
\end{figure}

\begin{figure}[H]
\centering
\caption{Estimated Coefficients of Capital and Labor for US: NAICS 33}
\includegraphics[width=9cm, height=9cm]{/Users/justindoty/Documents/Research/Dissertation/Production_QR_Proxy/Code/Empirical/US/Plots/QACF_Coef_Plot_NAICS_33.png}
\label{fig:ACF33coef}
\end{figure}

\begin{figure}[H]
\centering
\caption{Estimated Coefficients of Capital and Labor for All US Manufacturing Firms}
\includegraphics[width=9cm, height=9cm]{/Users/justindoty/Documents/Research/Dissertation/Production_QR_Proxy/Code/Empirical/US/Plots/QACF_Coef_Plot_NAICS_All.png}
\label{fig:ACFUSallcoef}
\end{figure}

We use the estimates from the output elasticities to construct measures of returns to scale and capital intensity in Table \ref{USestACF}. Estimates for returns to scale are not statistically different from constant returns. Returns to scale decrease slightly when $\tau$ increases. We compare the results to those obtained by using ACF to estimate productivity in Table \ref{USestACF}. We also report estimates of capital intensity measured by the ratio of capital to labor elasticity for each quantile. Each industry has estimates of capital intensity that increases with $\tau$. This result is consistent with previous findings such as \cite{Holmes2008}, \cite{Kumar1999} and \cite{mert}.


\begin{table}[H]
\centering
\caption{Coefficient Estimates and Standard Errors for US Manufacturing Firms}
\begin{tabular}{cccccccccc}
  \hline\hline & & \multicolumn{2}{c}{Capital}  & \multicolumn{2}{c}{Labor} & \multicolumn{2}{c}{Returns to Scale} & \multicolumn{2}{c}{Capital Intensity}\\ \cmidrule(lr){3-4} \cmidrule(lr){5-6} \cmidrule(lr){7-8} \cmidrule(lr){9-10}NAICS & $\tau$ & Coef. & s.e & Coef. & s.e & Coef. & s.e & Coef. & s.e \\ 
  \hline
31 & 0.10 & 0.142 & 0.0330 & 0.893 & 0.0754 & 1.035 & 0.0482 & 0.159 & 0.0437 \\ 
   & 0.25 & 0.179 & 0.0323 & 0.842 & 0.0740 & 1.021 & 0.0476 & 0.213 & 0.0487 \\ 
   & 0.50 & 0.225 & 0.0342 & 0.776 & 0.0765 & 1.001 & 0.0482 & 0.290 & 0.0608 \\ 
   & 0.90 & 0.281 & 0.0361 & 0.711 & 0.0771 & 0.992 & 0.0473 & 0.395 & 0.0795 \\ 
  32 & 0.10 & 0.101 & 0.0153 & 0.977 & 0.0303 & 1.078 & 0.0192 & 0.103 & 0.0172 \\ 
   & 0.25 & 0.127 & 0.0149 & 0.945 & 0.0286 & 1.071 & 0.0181 & 0.134 & 0.0182 \\ 
   & 0.50 & 0.159 & 0.0155 & 0.900 & 0.0293 & 1.059 & 0.0182 & 0.176 & 0.0208 \\ 
   & 0.90 & 0.220 & 0.0172 & 0.817 & 0.0318 & 1.037 & 0.0187 & 0.270 & 0.0286 \\ 
  33 & 0.10 & 0.025 & 0.0222 & 1.056 & 0.0451 & 1.082 & 0.0257 & 0.024 & 0.0230 \\ 
   & 0.25 & 0.084 & 0.0180 & 0.974 & 0.0319 & 1.059 & 0.0179 & 0.087 & 0.0216 \\ 
   & 0.50 & 0.127 & 0.0175 & 0.919 & 0.0287 & 1.046 & 0.0155 & 0.138 & 0.0237 \\ 
   & 0.90 & 0.191 & 0.0172 & 0.850 & 0.0261 & 1.041 & 0.0134 & 0.225 & 0.0274 \\ 
  All & 0.10 & 0.076 & 0.0134 & 1.003 & 0.0249 & 1.078 & 0.0139 & 0.075 & 0.0150 \\ 
   & 0.25 & 0.130 & 0.0119 & 0.926 & 0.0202 & 1.056 & 0.0111 & 0.140 & 0.0154 \\ 
   & 0.50 & 0.168 & 0.0118 & 0.876 & 0.0192 & 1.044 & 0.0104 & 0.192 & 0.0171 \\ 
   & 0.90 & 0.232 & 0.0127 & 0.801 & 0.0198 & 1.033 & 0.0101 & 0.290 & 0.0222 \\ 
   \hline
\end{tabular}
\caption*{\footnotesize $^{*}$Standard errors are obtained using bootstrap with 500 replications. Productivity is estimated using ACF}
\label{USestACF}
\end{table}

We also use our quantile production function estimates to construct measures of firm level productivity which we define as
\begin{equation}
TFP_{it}(\tau)=\exp(y_{it}-\hat{\beta_{k}}(\tau)k_{it}-\hat{\beta_{l}}(\tau)l_{it})
\end{equation}
We use these measures to compare productivity growth over time to ACF estimates over the distribution of firm size. Figure \ref{fig:ACFUSpgrowth} reports average productivity for all US firms in the sample with the base year of the sample period set to 100. We can see that productivity growth was rapid in the beginning of the sample period but then declined after 1970, but then increases after 1980. Growth trends for each percentile of firm size were similar although larger firms in this sample were more productive than smaller ones. The ACF estimates are close to the productivity estimates for medium-sized firms at $\tau=0.5$.

We are also interested in examining the firm size distribution over time and whether there are differences in factor intensities over time. Figure \ref{fig:ACFUStimecoef} plots estimates of the output elasticities over 5 year intervals. The panel on the left are the estimates of capital elasticity. These are similar for the beginning of the sample period and increase sharply. Dispersion between the estimates at $\tau=0.9$ and $\tau=0.1$ become quite high after 1970. and for each $\tau$ these estimates are decreasing in magnitude. After this decade, these estimates increase, but remain disperse. Estimates are increasing in $\tau$. The panel on the right are estimates of labor elasticity. In the beginning of the sample period, the estimates range from 0.7 to 0.85. There is a sharp increase, then a steady decrease after 1970. Estimates are decreasing in $\tau$.

\begin{figure}[H]
\centering
\caption{US Productivity Over Time}
\includegraphics[width=10cm]{/Users/justindoty/Documents/Research/Dissertation/Production_QR_Proxy/Code/Empirical/US/Plots/ACF_TFP_Plot.png}
\label{fig:ACFUSpgrowth}
\end{figure}

\begin{figure}[H]
\centering
\caption{Estimates of Capital and Labor Elasticities Over Time}
\includegraphics[height=5cm, width=10cm]{/Users/justindoty/Documents/Research/Dissertation/Production_QR_Proxy/Code/Empirical/US/Plots/QACF_Time_Plot.png}
\label{fig:ACFUStimecoef}
\end{figure}


%----------------------------------------------------------------------------------------

\subsection{Chilean Manufacturing}
This data comes from the census of Chilean manufacturing plants conducted by the Instituto Nacional de Estadist\'ica (INE). The sample is collected between 1979 and 1996 for firms with more than 10 employees. We divide our estimates into the three largest manufacturing industries: Food (ISIC 311), Fabricated Metals (ISIC 381), and Textiles (ISIC 321). We also aggregate the three industries with the other smaller industries to obtain estimates from the entire sample. Summary statistics for the data we use are provided in Table \ref{CHLsum}.

Figures \ref{fig:ACFCHL311}, \ref{fig:ACFCHL381} and \ref{fig:ACFCHL321} illustrate our estimates from our model compared to ACF estimates as well as their differences to QR estimates. In each industry and the combined sample, estimates of capital elasticity are increasing in firm size. In ISIC 311 and ISIC 381, our model estimates lower capital elasticities than the ACF estimate for small $\tau$. In ISIC 321 and the combined sample, there are differences between these estimates for both small and large $\tau$. The labor estimates in ISIC 311 are increasing but not different from the the ACF estimator. In ISIC 381, labor estimates are are mostly constant and not different from the ACF estimator. In ISIC 381, labor estimates are decreasing but not different from the ACF estimator. Finally, in the combined sample, our estimator is mostly constant and not different from the ACF estimator. Comparing our estimator and QR estimates, we find that there are differences in the estimates for capital and some evidence of differences in labor for small $\tau$ in ISIC 311, 381, and the combined sample. From Table \ref{CHLestACF}, returns to scale are increasing with firm size. Capital intensity is increasing as well except for large firms at $\tau=0.9$ in ISIC 311.

Figure \ref{fig:ACFCHLpgrowth} reports average productivity for all Chilean plants in the sample with base period set to 100. Productivity decreases in the beginning of the 1980s but then increases for the rest of the sample period. These estimates suggest that small firms had higher productivity than large firms. The ACF estimates are similar to productivity of medium-sized firms at $\tau=0.5$. Figure \ref{fig:ACFCHLtimecoef} shows the time trends in output elasticities. The estimates for capital and labor elasticities follow similar trends for each quantile. Capital estimates decrease in the beginning of the sample period then increase, reaching their peak before 1990. These estimates increase in $\tau$. Labor estimates are not much different across quantiles and decrease over time.

\begin{figure}[H]
\centering
\caption{Estimated Coefficients of Capital and Labor for Chile: ISIC 311}
\includegraphics[width=9cm, height=9cm]{/Users/justindoty/Documents/Research/Dissertation/Production_QR_Proxy/Code/Empirical/Chile/Plots/QACF_Coef_Plot_ISIC_311.png}
\label{fig:ACFCHL311}
\end{figure}

\begin{figure}[H]
\centering
\caption{Estimated Coefficients of Capital and Labor for Chile: ISIC 381}
\includegraphics[width=9cm, height=9cm]{/Users/justindoty/Documents/Research/Dissertation/Production_QR_Proxy/Code/Empirical/Chile/Plots/QACF_Coef_Plot_ISIC_381.png}
\label{fig:ACFCHL381}
\end{figure}

\begin{figure}[H]
\centering
\caption{Estimated Coefficients of Capital and Labor for Chile: ISIC 321}
\includegraphics[width=9cm, height=9cm]{/Users/justindoty/Documents/Research/Dissertation/Production_QR_Proxy/Code/Empirical/Chile/Plots/QACF_Coef_Plot_ISIC_321.png}
\label{fig:ACFCHL321}
\end{figure}

\begin{figure}[H]
\centering
\caption{Estimated Coefficients of Capital and Labor for all Chilean Manufacturing Plants}
\includegraphics[width=9cm, height=9cm]{/Users/justindoty/Documents/Research/Dissertation/Production_QR_Proxy/Code/Empirical/Chile/Plots/QACF_Coef_Plot_ISIC_All.png}
\label{fig:ACFCHLall}
\end{figure}

\begin{table}[H]
\centering
\caption{Coefficient Estimates and Standard Errors for Chilean Manufacturing Plants}
\begin{tabular}{cccccccccc}
  \hline\hline & & \multicolumn{2}{c}{Capital}  & \multicolumn{2}{c}{Labor} & \multicolumn{2}{c}{Returns to Scale} & \multicolumn{2}{c}{Capital Intensity}\\ \cmidrule(lr){3-4} \cmidrule(lr){5-6} \cmidrule(lr){7-8} \cmidrule(lr){9-10}ISIC & $\tau$ & Coef. & s.e & Coef. & s.e & Coef. & s.e & Coef. & s.e \\ 
  \hline
311 & 0.10 & 0.147 & 0.0211 & 0.721 & 0.0463 & 0.868 & 0.0348 & 0.204 & 0.0408 \\ 
   & 0.25 & 0.199 & 0.0187 & 0.761 & 0.0394 & 0.960 & 0.0315 & 0.262 & 0.0351 \\ 
   & 0.50 & 0.251 & 0.0179 & 0.776 & 0.0342 & 1.028 & 0.0282 & 0.324 & 0.0333 \\ 
   & 0.90 & 0.296 & 0.0195 & 0.898 & 0.0417 & 1.194 & 0.0324 & 0.330 & 0.0343 \\ 
  381 & 0.10 & 0.145 & 0.0294 & 0.920 & 0.0611 & 1.065 & 0.0418 & 0.157 & 0.0411 \\ 
   & 0.25 & 0.216 & 0.0253 & 0.880 & 0.0510 & 1.096 & 0.0366 & 0.245 & 0.0403 \\ 
   & 0.50 & 0.275 & 0.0253 & 0.846 & 0.0505 & 1.121 & 0.0360 & 0.325 & 0.0464 \\ 
   & 0.90 & 0.354 & 0.0284 & 0.848 & 0.0557 & 1.201 & 0.0375 & 0.417 & 0.0568 \\ 
  321 & 0.10 & 0.119 & 0.0338 & 0.940 & 0.0682 & 1.058 & 0.0442 & 0.126 & 0.0435 \\ 
   & 0.25 & 0.186 & 0.0281 & 0.915 & 0.0527 & 1.101 & 0.0367 & 0.204 & 0.0420 \\ 
   & 0.50 & 0.270 & 0.0273 & 0.831 & 0.0505 & 1.101 & 0.0358 & 0.324 & 0.0498 \\ 
   & 0.90 & 0.357 & 0.0271 & 0.804 & 0.0552 & 1.161 & 0.0401 & 0.445 & 0.0614 \\ 
  All & 0.10 & 0.194 & 0.0115 & 0.828 & 0.0233 & 1.022 & 0.0155 & 0.234 & 0.0198 \\ 
   & 0.25 & 0.268 & 0.0098 & 0.810 & 0.0181 & 1.078 & 0.0128 & 0.330 & 0.0182 \\ 
   & 0.50 & 0.339 & 0.0091 & 0.771 & 0.0154 & 1.110 & 0.0111 & 0.439 & 0.0190 \\ 
   & 0.90 & 0.418 & 0.0103 & 0.806 & 0.0190 & 1.224 & 0.0129 & 0.519 & 0.0237 \\ 
   \hline
\end{tabular}
\caption*{\footnotesize $^{*}$Standard errors are obtained using bootstrap with 500 replications. Productivity is estimated using ACF}
\label{CHLestACF}
\end{table}

\begin{figure}[H]
\centering
\caption{Chile Productivity Over Time}
\includegraphics[width=10cm]{/Users/justindoty/Documents/Research/Dissertation/Production_QR_Proxy/Code/Empirical/Chile/Plots/ACF_TFP_Plot.png}
\label{fig:ACFCHLpgrowth}
\end{figure}

\begin{figure}[H]
\centering
\caption{Estimates of Capital and Labor Elasticities Over Time}
\includegraphics[height=5cm, width=10cm]{/Users/justindoty/Documents/Research/Dissertation/Production_QR_Proxy/Code/Empirical/Chile/Plots/QACF_Time_Plot.png}
\label{fig:ACFCHLtimecoef}
\end{figure}


%------------------------------------------------------------------------------------------------

\subsection{Colombian Manufacturing}
This data comes from the Colombian manufacturing census conducted by the Departamento Administrativo Nacional de Estadistica. The sample is collected between 1977 and 1991. We divide our estimates into the three largest manufacturing industries: Food (ISIC 311), Apparel (ISIC 322), and Fabricated Metals (ISIC 381). As we did with the Chilean sample, we also aggregate the three industries with other smaller industries to obtain estimates from the entire sample of manufacturing plants. Summary statistics for this data is provided in Table \ref{COLsum}.

Figures \ref{fig:ACFCOL311}, \ref{fig:ACFCOL321}, and \ref{fig:ACFCOL381} illustrate estimates from our model compared to the ACF estimates as well as their differences from QR estimates. The capital estimates in each industry are increasing and are significantly different than the ACF estimates. The labor estimates in each industry are decreasing, however, only in the combined sample are these estimates different from ACF for both low and high $\tau$. Comparing our results to those obtained using QR, there is statistical difference in ISIC 311, difference in the labor estimate in ISIC 321 and ISIC 381 and differences in both capital and labor in the combined sample.

\begin{figure}[H]
\centering
\caption{Estimated Coefficients of Capital and Labor for Colombia: ISIC 311}
\includegraphics[width=9cm, height=9cm]{/Users/justindoty/Documents/Research/Dissertation/Production_QR_Proxy/Code/Empirical/Colombia/Plots/QACF_Coef_Plot_ISIC_311.png}
\label{fig:ACFCOL311}
\end{figure}

\begin{figure}[H]
\centering
\caption{Estimated Coefficients of Capital and Labor for Colombia: ISIC 321}
\includegraphics[width=9cm, height=9cm]{/Users/justindoty/Documents/Research/Dissertation/Production_QR_Proxy/Code/Empirical/Colombia/Plots/QACF_Coef_Plot_ISIC_322.png}
\label{fig:ACFCOL321}
\end{figure}

\begin{figure}[H]
\centering
\caption{Estimated Coefficients of Capital and Labor for Colombia: ISIC 381}
\includegraphics[width=9cm, height=9cm]{/Users/justindoty/Documents/Research/Dissertation/Production_QR_Proxy/Code/Empirical/Colombia/Plots/QACF_Coef_Plot_ISIC_381.png}
\label{fig:ACFCOL381}
\end{figure}

\begin{figure}[H]
\centering
\caption{Estimated Coefficients of Capital and Labor for all Colombian Manufacturing Plants}
\includegraphics[width=9cm, height=9cm]{/Users/justindoty/Documents/Research/Dissertation/Production_QR_Proxy/Code/Empirical/Colombia/Plots/QACF_Coef_Plot_ISIC_All.png}
\label{fig:ACFCOLall}
\end{figure}

Using these estimates we construct measures of returns to scale and capital intensity for each industry in Table \ref{COLestACF}. Most firms experience constant or slightly decreasing returns to scale. Returns to scale are smallest in ISIC 311. Returns to scale and capital intensity are increasing in firm size.

\begin{table}[H]
\centering
\caption{Coefficient Estimates and Standard Errors for Colombian Manufacturing Plants}
\begin{tabular}{cccccccccc}
  \hline\hline & & \multicolumn{2}{c}{Capital}  & \multicolumn{2}{c}{Labor} & \multicolumn{2}{c}{Returns to Scale} & \multicolumn{2}{c}{Capital Intensity}\\ \cmidrule(lr){3-4} \cmidrule(lr){5-6} \cmidrule(lr){7-8} \cmidrule(lr){9-10}ISIC & $\tau$ & Coef. & s.e & Coef. & s.e & Coef. & s.e & Coef. & s.e \\ 
  \hline
311 & 0.10 & 0.085 & 0.0217 & 0.649 & 0.0505 & 0.735 & 0.0448 & 0.131 & 0.0376 \\ 
   & 0.25 & 0.156 & 0.0201 & 0.622 & 0.0448 & 0.777 & 0.0419 & 0.251 & 0.0395 \\ 
   & 0.50 & 0.231 & 0.0201 & 0.563 & 0.0438 & 0.795 & 0.0412 & 0.411 & 0.0503 \\ 
   & 0.90 & 0.398 & 0.0230 & 0.423 & 0.0496 & 0.822 & 0.0429 & 0.942 & 0.1251 \\ 
  322 & 0.10 & 0.048 & 0.0283 & 0.583 & 0.0444 & 0.631 & 0.0413 & 0.083 & 0.0504 \\ 
   & 0.25 & 0.108 & 0.0279 & 0.576 & 0.0432 & 0.683 & 0.0407 & 0.187 & 0.0549 \\ 
   & 0.50 & 0.197 & 0.0279 & 0.508 & 0.0422 & 0.705 & 0.0401 & 0.389 & 0.0718 \\ 
   & 0.90 & 0.364 & 0.0296 & 0.382 & 0.0428 & 0.746 & 0.0394 & 0.951 & 0.1509 \\ 
  381 & 0.10 & 0.151 & 0.0363 & 0.904 & 0.0589 & 1.055 & 0.0472 & 0.167 & 0.0487 \\ 
   & 0.25 & 0.240 & 0.0329 & 0.863 & 0.0426 & 1.102 & 0.0401 & 0.278 & 0.0472 \\ 
   & 0.50 & 0.341 & 0.0345 & 0.763 & 0.0435 & 1.104 & 0.0391 & 0.447 & 0.0640 \\ 
   & 0.90 & 0.478 & 0.0385 & 0.680 & 0.0498 & 1.159 & 0.0389 & 0.703 & 0.0975 \\ 
  All & 0.10 & 0.105 & 0.0101 & 0.737 & 0.0214 & 0.843 & 0.0186 & 0.143 & 0.0156 \\ 
   & 0.25 & 0.189 & 0.0095 & 0.710 & 0.0198 & 0.899 & 0.0179 & 0.266 & 0.0173 \\ 
   & 0.50 & 0.268 & 0.0096 & 0.660 & 0.0197 & 0.928 & 0.0177 & 0.406 & 0.0215 \\ 
   & 0.90 & 0.417 & 0.0105 & 0.571 & 0.0209 & 0.988 & 0.0179 & 0.730 & 0.0368 \\ 
   \hline
\end{tabular}
\caption*{\footnotesize $^{*}$Standard errors are obtained using bootstrap with 500 replications. Productivity is estimated using ACF}
\label{COLestACF}
\end{table}

Figure \ref{fig:ACFCOACFgrowth} reports average productivity for all Colombian plants in the sample with base period set to 100. Productivity decreases in the beginning of the sample period, but then increases for the rest of the sample period after 1980 with some periods of sharp changes. Each percentile of firm size has similar productivity levels throughout the sample period. Figure \ref{fig:ACFCOLtimecoef} shows the time trends in output elasticities. Capital estimates are generally increases except for a slight dip in 1985. These estimates are increasing in $\tau$. The estimates of labor elasticity are similar in the beginning of the sample period. They increase past 1980 where dispersion across quantiles is largest, then decrease and dispersion lessens.

\begin{figure}[H]
\centering
\caption{Colombia Productivity Over Time}
\includegraphics[width=10cm]{/Users/justindoty/Documents/Research/Dissertation/Production_QR_Proxy/Code/Empirical/Colombia/Plots/ACF_TFP_Plot.png}
\label{fig:ACFCOLpgrowth}
\end{figure}

\begin{figure}[H]
\centering
\caption{Estimates of Capital Labor Elasticities Over Time}
\includegraphics[height=5cm, width=10cm]{/Users/justindoty/Documents/Research/Dissertation/Production_QR_Proxy/Code/Empirical/Colombia/Plots/QACF_Time_Plot.png}
\label{fig:ACFCOLtimecoef}
\end{figure}

\section{Conclusions} \label{conclusion}

We propose a method that extends the proxy variable approach to estimating the conditional quantiles of firm size. The method is computationally attractive as it resembles the two-stage estimator proposed by \cite{Canay2011}. As a result, practitioners are able to easily apply the proposed estimator to production function models where the data may reveal significant heterogeneous output elasticities along the conditional firm size distribution. We showed that this estimator works well in finite samples and showed that it captures heterogeneity in firm size under different data generating processes. An application to widely used datasets from the US, Chile, and Colombia showed that in some industries, our estimator captures unobserved heterogeneity that the LP estimator does not.

Improvements and extensions of this estimator are currently being explored. For example, using a value-added production function may show more heterogeneity in estimates of elasticities and productivity than a gross-output production function. However, using a gross-output production function with an intermediate input proxy variable suffers from non-identification. We are currently working on an identification strategy that allows for different types of production function estimators such as the one proposed by \cite{Gandhi2020}. This paper also makes an interesting but brief connection between the literature on production risk and quantile utility maximization. Currently, quantile utility maximization problems and estimation of these models are being studied by \cite{Castro2017} and \cite{qgmm} in the context of dynamic consumption problems. It would be interesting to explore a model for a firm who maximizes quantile utility of profits which could provide an alternative interpretation for unobserved heterogeneity that are obtained from quantile regression estimates.

This paper contributes to the growing literature on production functions with unobserved heterogeneity. We show that differences in firm size correspond to the the rank of the ex-post shock. The location-shift model for productivity we propose here restricts us from examining other dimensions of firm heterogeneity. Therefore, allowing richer distributional effects of productivity would be an interesting extension. This approach also restricts us from examining non-Hicks neutral productivity shocks such as factor-augmenting productivity. We are currently working on an extension of this paper to a non-separable model to address these last two points, but the estimator we propose here is computationally attractive and easy to implement in empirical research.    




\pagebreak
\newpage



%\section*{Appendix}
%\appendix
%\begin{Large}
%\noindent \textbf{A. Proof of the Theorems}
%\end{Large}

%\counterwithin{theorem}{section}
%\section{Proofs}





\bibliographystyle{ecca.bst}
\bibliography{references}

\section*{Appendix} \label{Appendix}
\subsection{Summary Statistics}
\subsubsection{US}
\begin{table}[H]
\centering
\caption{Summary Statistics (in logs) for US Manufacturing Data}
\begin{tabular}{ccccccc}
  \hline\hline Industry (NAICS code) &   & 1st Qu. & Median & 3rd Qu. & Mean & sd \\ 
  \hline
31 (Total=3271) & Output & 19.05 & 20.24 & 21.57 & 20.3 & 1.77 \\ 
   & Capital & 18.66 & 20.37 & 21.76 & 20.19 & 2.12 \\ 
   & Labor & 17.42 & 19.08 & 20.61 & 19.02 & 2.21 \\ 
   & Materials & 17.96 & 19.59 & 21.15 & 19.54 & 2.21 \\ 
  32 (Total=7207) & Output & 15.67 & 17.04 & 18.51 & 17.01 & 2.05 \\ 
   & Capital & 15.65 & 17.51 & 19.13 & 17.31 & 2.41 \\ 
   & Labor & 14.44 & 16.01 & 17.57 & 16.01 & 2.29 \\ 
   & Materials & 14.89 & 16.53 & 18.25 & 16.52 & 2.37 \\ 
  33 (Total=13978) & Output & 7.38 & 8.58 & 9.8 & 8.5 & 1.67 \\ 
   & Capital & 6.67 & 8.29 & 9.74 & 8.15 & 1.95 \\ 
   & Labor & 6.01 & 7.42 & 8.91 & 7.48 & 1.93 \\ 
   & Materials & 6.33 & 7.82 & 9.29 & 7.82 & 1.95 \\ 
  All (Total=24456) & Output & 18.58 & 19.78 & 21.23 & 19.85 & 1.79 \\ 
   & Capital & 18.14 & 19.86 & 21.26 & 19.67 & 2.16 \\ 
   & Labor & 16.98 & 18.59 & 20.13 & 18.56 & 2.17 \\ 
   & Materials & 17.49 & 19.12 & 20.66 & 19.06 & 2.2 \\ 
   \hline
\end{tabular}
\label{USsum}
\end{table}

\subsubsection{Chile}

\begin{table}[H]
\centering
\caption{Summary Statistics (in logs) for Chilean Manufacturing Plants}
\begin{tabular}{ccccccc}
  \hline\hline Industry (ISIC code) &   & 1st Qu. & Median & 3rd Qu. & Mean & sd \\ 
  \hline
311 (Total=13838) & Output & 10.21 & 10.84 & 12.22 & 11.36 & 1.58 \\ 
   & Capital & 10.56 & 11.4 & 12.4 & 11.52 & 1.37 \\ 
   & Labor & 10.49 & 11.4 & 12.54 & 11.53 & 1.43 \\ 
   & Materials & 10.38 & 11.28 & 12.53 & 11.56 & 1.6 \\ 
  381 (Total=4311) & Output & 6.69 & 7.66 & 9.06 & 8.02 & 1.98 \\ 
   & Capital & 7.52 & 8.51 & 9.7 & 8.65 & 1.68 \\ 
   & Labor & 7.21 & 8.34 & 9.56 & 8.4 & 1.72 \\ 
   & Materials & 7.22 & 8.35 & 9.72 & 8.54 & 1.92 \\ 
  321 (Total=4302) & Output & 2.77 & 3.22 & 3.91 & 3.49 & 0.99 \\ 
   & Capital & 2.89 & 3.47 & 4.22 & 3.71 & 1.08 \\ 
   & Labor & 2.94 & 3.48 & 4.37 & 3.69 & 0.95 \\ 
   & Materials & 2.89 & 3.43 & 4.28 & 3.67 & 1.02 \\ 
  All (Total=51567) & Output & 9.84 & 10.46 & 11.81 & 10.94 & 1.56 \\ 
   & Capital & 9.91 & 10.75 & 11.79 & 10.86 & 1.41 \\ 
   & Labor & 9.68 & 10.62 & 11.75 & 10.73 & 1.48 \\ 
   & Materials & 9.81 & 10.68 & 11.89 & 10.93 & 1.62 \\ 
   \hline
\end{tabular}
\label{CHLsum}
\end{table} 

\subsubsection{Colombia}

\begin{table}[H]
\centering
\caption{Summary Statistics (in logs) for Colombian Manufacturing Plants}
\begin{tabular}{ccccccc}
  \hline\hline Industry (ISIC code) &   & 1st Qu. & Median & 3rd Qu. & Mean & sd \\ 
  \hline
311 (Total=13215) & Output & 9.03 & 10.21 & 11.59 & 10.42 & 1.8 \\ 
   & Capital & 8.69 & 9.37 & 10.22 & 9.49 & 1.18 \\ 
   & Labor & 8.52 & 9.3 & 10.33 & 9.54 & 1.43 \\ 
   & Materials & 8.7 & 9.62 & 10.88 & 9.92 & 1.67 \\ 
  322 (Total=12182) & Output & 6.02 & 7.07 & 8.35 & 7.24 & 1.78 \\ 
   & Capital & 5.47 & 6.14 & 6.93 & 6.23 & 1.21 \\ 
   & Labor & 5.89 & 6.75 & 7.81 & 6.93 & 1.55 \\ 
   & Materials & 5.9 & 6.89 & 8.16 & 7.12 & 1.77 \\ 
  381 (Total=7411) & Output & 2.56 & 3.09 & 3.97 & 3.36 & 1.1 \\ 
   & Capital & 2.77 & 3.3 & 3.95 & 3.42 & 0.92 \\ 
   & Labor & 2.64 & 3.18 & 3.91 & 3.37 & 0.98 \\ 
   & Materials & 2.71 & 3.3 & 4.11 & 3.5 & 1.09 \\ 
  All (Total=87783) & Output & 8.39 & 9.73 & 11.26 & 9.87 & 2 \\ 
   & Capital & 7.62 & 8.53 & 9.46 & 8.48 & 1.51 \\ 
   & Labor & 7.77 & 8.65 & 9.72 & 8.8 & 1.58 \\ 
   & Materials & 7.89 & 8.93 & 10.26 & 9.15 & 1.88 \\ 
   \hline
\end{tabular}
\label{COLsum}
\end{table}

\subsection{Estimates using \cite{Levinsohn2003}}

\subsubsection{US}

\begin{table}[H]
\centering
\caption{Coefficient Estimates and Standard Errors for US Manufacturing Firms}
\begin{tabular}{cccccccccc}
  \hline\hline & & \multicolumn{2}{c}{Capital}  & \multicolumn{2}{c}{Labor} & \multicolumn{2}{c}{Returns to Scale} & \multicolumn{2}{c}{Capital Intensity}\\ \cmidrule(lr){3-4} \cmidrule(lr){5-6} \cmidrule(lr){7-8} \cmidrule(lr){9-10}NAICS & $\tau$ & Coef. & s.e & Coef. & s.e & Coef. & s.e & Coef. & s.e \\ 
  \hline
31 & 0.10 & 0.054 & 0.0174 & 0.601 & 0.0269 & 0.654 & 0.0318 & 0.089 & 0.0296 \\ 
   & 0.25 & 0.085 & 0.0182 & 0.560 & 0.0302 & 0.645 & 0.0329 & 0.152 & 0.0355 \\ 
   & 0.50 & 0.138 & 0.0198 & 0.480 & 0.0325 & 0.618 & 0.0332 & 0.288 & 0.0499 \\ 
   & 0.90 & 0.185 & 0.0223 & 0.433 & 0.0375 & 0.618 & 0.0344 & 0.428 & 0.0830 \\ 
  32 & 0.10 & 0.051 & 0.0109 & 0.661 & 0.0306 & 0.713 & 0.0277 & 0.077 & 0.0181 \\ 
   & 0.25 & 0.079 & 0.0102 & 0.626 & 0.0279 & 0.705 & 0.0264 & 0.126 & 0.0188 \\ 
   & 0.50 & 0.112 & 0.0098 & 0.581 & 0.0265 & 0.692 & 0.0259 & 0.192 & 0.0207 \\ 
   & 0.90 & 0.177 & 0.0127 & 0.490 & 0.0298 & 0.667 & 0.0265 & 0.361 & 0.0415 \\ 
  33 & 0.10 & -0.009 & 0.0224 & 0.438 & 0.0608 & 0.429 & 0.0459 & -0.020 & 0.0667 \\ 
   & 0.25 & 0.042 & 0.0162 & 0.376 & 0.0360 & 0.419 & 0.0322 & 0.112 & 0.0573 \\ 
   & 0.50 & 0.086 & 0.0158 & 0.319 & 0.0331 & 0.405 & 0.0305 & 0.270 & 0.0796 \\ 
   & 0.90 & 0.167 & 0.0163 & 0.214 & 0.0293 & 0.381 & 0.0278 & 0.782 & 0.1859 \\ 
  All & 0.10 & 0.017 & 0.0107 & 0.506 & 0.0292 & 0.523 & 0.0259 & 0.033 & 0.0239 \\ 
   & 0.25 & 0.067 & 0.0099 & 0.439 & 0.0244 & 0.506 & 0.0233 & 0.153 & 0.0292 \\ 
   & 0.50 & 0.104 & 0.0098 & 0.393 & 0.0235 & 0.497 & 0.0228 & 0.263 & 0.0369 \\ 
   & 0.90 & 0.176 & 0.0106 & 0.300 & 0.0222 & 0.476 & 0.0215 & 0.587 & 0.0713 \\ 
   \hline
\end{tabular}
\caption*{\footnotesize $^{*}$Standard errors are obtained using bootstrap with 500 replications. Productivity is estimated using LP}
\label{USestLP}
\end{table}

\begin{figure}[H]
\centering
\caption{Estimated Coefficients of Capital and Labor for US: NAICS 31}
\includegraphics[width=9cm, height=9cm]{/Users/justindoty/Documents/Research/Dissertation/Production_QR_Proxy/Code/Empirical/US/Plots/QLP_Coef_Plot_NAICS_31.png}
\label{fig:LP31coef}
\end{figure}

\begin{figure}[H]
\centering
\caption{Estimated Coefficients of Capital and Labor for US: NAICS 32}
\includegraphics[width=9cm, height=9cm]{/Users/justindoty/Documents/Research/Dissertation/Production_QR_Proxy/Code/Empirical/US/Plots/QLP_Coef_Plot_NAICS_32.png}
\label{fig:LP32coef}
\end{figure}

\begin{figure}[H]
\centering
\caption{Estimated Coefficients of Capital and Labor for US: NAICS 33}
\includegraphics[width=9cm, height=9cm]{/Users/justindoty/Documents/Research/Dissertation/Production_QR_Proxy/Code/Empirical/US/Plots/QLP_Coef_Plot_NAICS_33.png}
\label{fig:LP33coef}
\end{figure}

\begin{figure}[H]
\centering
\caption{Estimated Coefficients of Capital and Labor for All US Manufacturing Firms}
\includegraphics[width=9cm, height=9cm]{/Users/justindoty/Documents/Research/Dissertation/Production_QR_Proxy/Code/Empirical/US/Plots/QLP_Coef_Plot_NAICS_All.png}
\label{fig:LPUSallcoef}
\end{figure}

\begin{figure}[H]
\centering
\caption{US Productivity Over Time}
\includegraphics[width=10cm]{/Users/justindoty/Documents/Research/Dissertation/Production_QR_Proxy/Code/Empirical/US/Plots/LP_TFP_Plot.png}
\label{fig:LPUSpgrowth}
\end{figure}

\begin{figure}[H]
\centering
\caption{Estimates of Capital and Labor Elasticities Over Time}
\includegraphics[height=5cm, width=10cm]{/Users/justindoty/Documents/Research/Dissertation/Production_QR_Proxy/Code/Empirical/US/Plots/QLP_Time_Plot.png}
\label{fig:LPUStimecoef}
\end{figure}

\subsubsection{Chile}

\begin{table}[H]
\centering
\caption{Coefficient Estimates and Standard Errors for Chilean Manufacturing Plants}
\begin{tabular}{cccccccccc}
  \hline\hline & & \multicolumn{2}{c}{Capital}  & \multicolumn{2}{c}{Labor} & \multicolumn{2}{c}{Returns to Scale} & \multicolumn{2}{c}{Capital Intensity}\\ \cmidrule(lr){3-4} \cmidrule(lr){5-6} \cmidrule(lr){7-8} \cmidrule(lr){9-10}ISIC & $\tau$ & Coef. & s.e & Coef. & s.e & Coef. & s.e & Coef. & s.e \\ 
  \hline
311 & 0.10 & 0.195 & 0.0225 & 0.356 & 0.0365 & 0.552 & 0.0304 & 0.548 & 0.1100 \\ 
   & 0.25 & 0.245 & 0.0204 & 0.403 & 0.0271 & 0.648 & 0.0265 & 0.607 & 0.0767 \\ 
   & 0.50 & 0.302 & 0.0195 & 0.406 & 0.0219 & 0.707 & 0.0243 & 0.744 & 0.0720 \\ 
   & 0.90 & 0.335 & 0.0202 & 0.551 & 0.0326 & 0.887 & 0.0300 & 0.608 & 0.0610 \\ 
  381 & 0.10 & 0.232 & 0.0299 & 0.651 & 0.0583 & 0.883 & 0.0442 & 0.356 & 0.0771 \\ 
   & 0.25 & 0.305 & 0.0251 & 0.606 & 0.0424 & 0.911 & 0.0371 & 0.503 & 0.0673 \\ 
   & 0.50 & 0.372 & 0.0248 & 0.553 & 0.0382 & 0.926 & 0.0345 & 0.673 & 0.0791 \\ 
   & 0.90 & 0.437 & 0.0293 & 0.586 & 0.0536 & 1.023 & 0.0410 & 0.745 & 0.1118 \\ 
  321 & 0.10 & 0.156 & 0.0358 & 0.709 & 0.0638 & 0.864 & 0.0462 & 0.220 & 0.0687 \\ 
   & 0.25 & 0.239 & 0.0301 & 0.648 & 0.0444 & 0.887 & 0.0378 & 0.369 & 0.0659 \\ 
   & 0.50 & 0.319 & 0.0291 & 0.573 & 0.0417 & 0.892 & 0.0368 & 0.556 & 0.0824 \\ 
   & 0.90 & 0.404 & 0.0300 & 0.554 & 0.0452 & 0.957 & 0.0386 & 0.729 & 0.1073 \\ 
  All & 0.10 & 0.222 & 0.0113 & 0.500 & 0.0227 & 0.722 & 0.0170 & 0.444 & 0.0393 \\ 
   & 0.25 & 0.301 & 0.0100 & 0.466 & 0.0171 & 0.767 & 0.0139 & 0.645 & 0.0400 \\ 
   & 0.50 & 0.371 & 0.0092 & 0.428 & 0.0138 & 0.799 & 0.0122 & 0.868 & 0.0427 \\ 
   & 0.90 & 0.439 & 0.0111 & 0.493 & 0.0196 & 0.932 & 0.0144 & 0.890 & 0.0542 \\ 
   \hline
\end{tabular}
\caption*{\footnotesize Standard errors are obtained using bootstrap with 500 replications. Productivity is estimated using LP.}
\label{CHLestLP}
\end{table}

\begin{figure}[H]
\centering
\caption{Estimated Coefficients of Capital and Labor for Chile: ISIC 311}
\includegraphics[width=9cm, height=9cm]{/Users/justindoty/Documents/Research/Dissertation/Production_QR_Proxy/Code/Empirical/Chile/Plots/QLP_Coef_Plot_ISIC_311.png}
\label{fig:LPCHL311}
\end{figure}

\begin{figure}[H]
\centering
\caption{Estimated Coefficients of Capital and Labor for Chile: ISIC 381}
\includegraphics[width=9cm, height=9cm]{/Users/justindoty/Documents/Research/Dissertation/Production_QR_Proxy/Code/Empirical/Chile/Plots/QLP_Coef_Plot_ISIC_381.png}
\label{fig:LPCHL381}
\end{figure}

\begin{figure}[H]
\centering
\caption{Estimated Coefficients of Capital and Labor for Chile: ISIC 321}
\includegraphics[width=9cm, height=9cm]{/Users/justindoty/Documents/Research/Dissertation/Production_QR_Proxy/Code/Empirical/Chile/Plots/QLP_Coef_Plot_ISIC_321.png}
\label{fig:LPCHL321}
\end{figure}

\begin{figure}[H]
\centering
\caption{Estimated Coefficients of Capital and Labor for all Chilean Manufacturing Plants}
\includegraphics[width=9cm, height=9cm]{/Users/justindoty/Documents/Research/Dissertation/Production_QR_Proxy/Code/Empirical/Chile/Plots/QLP_Coef_Plot_ISIC_All.png}
\label{fig:LPCHLall}
\end{figure}

\begin{figure}[H]
\centering
\caption{Chile Productivity Over Time}
\includegraphics[width=10cm]{/Users/justindoty/Documents/Research/Dissertation/Production_QR_Proxy/Code/Empirical/Chile/Plots/LP_TFP_Plot.png}
\label{fig:LPCHLpgrowth}
\end{figure}

\begin{figure}[H]
\centering
\caption{Estimates of Capital and Labor Elasticities Over Time}
\includegraphics[height=5cm, width=10cm]{/Users/justindoty/Documents/Research/Dissertation/Production_QR_Proxy/Code/Empirical/Chile/Plots/QLP_Time_Plot.png}
\label{fig:LPCHLtimecoef}
\end{figure}

\subsubsection{Colombia}

\begin{table}[H]
\centering
\caption{Coefficient Estimates and Standard Errors for Colombian Manufacturing Plants}
\begin{tabular}{cccccccccc}
  \hline\hline & & \multicolumn{2}{c}{Capital}  & \multicolumn{2}{c}{Labor} & \multicolumn{2}{c}{Returns to Scale} & \multicolumn{2}{c}{Capital Intensity}\\ \cmidrule(lr){3-4} \cmidrule(lr){5-6} \cmidrule(lr){7-8} \cmidrule(lr){9-10}ISIC & $\tau$ & Coef. & s.e & Coef. & s.e & Coef. & s.e & Coef. & s.e \\ 
  \hline
311 & 0.10 & 0.175 & 0.0278 & 0.618 & 0.0331 & 0.793 & 0.0351 & 0.284 & 0.0520 \\ 
   & 0.25 & 0.237 & 0.0265 & 0.612 & 0.0193 & 0.849 & 0.0299 & 0.387 & 0.0473 \\ 
   & 0.50 & 0.324 & 0.0266 & 0.530 & 0.0172 & 0.854 & 0.0291 & 0.612 & 0.0568 \\ 
   & 0.90 & 0.510 & 0.0281 & 0.354 & 0.0283 & 0.864 & 0.0320 & 1.439 & 0.1609 \\ 
  322 & 0.10 & 0.144 & 0.0242 & 0.683 & 0.0281 & 0.828 & 0.0306 & 0.212 & 0.0387 \\ 
   & 0.25 & 0.202 & 0.0250 & 0.676 & 0.0256 & 0.878 & 0.0286 & 0.299 & 0.0418 \\ 
   & 0.50 & 0.296 & 0.0260 & 0.604 & 0.0274 & 0.899 & 0.0285 & 0.490 & 0.0548 \\ 
   & 0.90 & 0.492 & 0.0322 & 0.441 & 0.0418 & 0.933 & 0.0313 & 1.116 & 0.1645 \\ 
  381 & 0.10 & 0.167 & 0.0385 & 0.933 & 0.0508 & 1.099 & 0.0501 & 0.179 & 0.0448 \\ 
   & 0.25 & 0.265 & 0.0356 & 0.871 & 0.0330 & 1.135 & 0.0442 & 0.304 & 0.0438 \\ 
   & 0.50 & 0.370 & 0.0358 & 0.763 & 0.0306 & 1.132 & 0.0430 & 0.485 & 0.0521 \\ 
   & 0.90 & 0.509 & 0.0397 & 0.678 & 0.0403 & 1.188 & 0.0432 & 0.751 & 0.0822 \\ 
  All & 0.10 & 0.097 & 0.0115 & 0.738 & 0.0136 & 0.836 & 0.0133 & 0.132 & 0.0169 \\ 
   & 0.25 & 0.183 & 0.0108 & 0.706 & 0.0096 & 0.889 & 0.0119 & 0.259 & 0.0169 \\ 
   & 0.50 & 0.266 & 0.0108 & 0.648 & 0.0085 & 0.915 & 0.0115 & 0.411 & 0.0190 \\ 
   & 0.90 & 0.413 & 0.0117 & 0.570 & 0.0117 & 0.983 & 0.0121 & 0.724 & 0.0306 \\ 
   \hline
\end{tabular}
\caption*{\footnotesize Standard errors are obtained using bootstrap with 500 replications. Productivity is estimated using LP.}
\label{COLestLP}
\end{table}

\begin{figure}[H]
\centering
\caption{Estimated Coefficients of Capital and Labor for Colombia: ISIC 311}
\includegraphics[width=9cm, height=9cm]{/Users/justindoty/Documents/Research/Dissertation/Production_QR_Proxy/Code/Empirical/Colombia/Plots/QLP_Coef_Plot_ISIC_311.png}
\label{fig:LPCOL311}
\end{figure}

\begin{figure}[H]
\centering
\caption{Estimated Coefficients of Capital and Labor for Colombia: ISIC 321}
\includegraphics[width=9cm, height=9cm]{/Users/justindoty/Documents/Research/Dissertation/Production_QR_Proxy/Code/Empirical/Colombia/Plots/QLP_Coef_Plot_ISIC_322.png}
\label{fig:LPCOL321}
\end{figure}

\begin{figure}[H]
\centering
\caption{Estimated Coefficients of Capital and Labor for Colombia: ISIC 381}
\includegraphics[width=9cm, height=9cm]{/Users/justindoty/Documents/Research/Dissertation/Production_QR_Proxy/Code/Empirical/Colombia/Plots/QLP_Coef_Plot_ISIC_381.png}
\label{fig:LPCOL381}
\end{figure}

\begin{figure}[H]
\centering
\caption{Estimated Coefficients of Capital and Labor for all Colombian Manufacturing Plants}
\includegraphics[width=9cm, height=9cm]{/Users/justindoty/Documents/Research/Dissertation/Production_QR_Proxy/Code/Empirical/Colombia/Plots/QLP_Coef_Plot_ISIC_All.png}
\label{fig:LPCOLall}
\end{figure}

\begin{figure}[H]
\centering
\caption{Colombia Productivity Over Time}
\includegraphics[width=10cm]{/Users/justindoty/Documents/Research/Dissertation/Production_QR_Proxy/Code/Empirical/Colombia/Plots/LP_TFP_Plot.png}
\label{fig:LPCOLpgrowth}
\end{figure}

\begin{figure}[H]
\centering
\caption{Estimates of Capital Labor Elasticities Over Time}
\includegraphics[height=5cm, width=10cm]{/Users/justindoty/Documents/Research/Dissertation/Production_QR_Proxy/Code/Empirical/Colombia/Plots/QLP_Time_Plot.png}
\label{fig:LPCOLtimecoef}
\end{figure}






















\end{document}